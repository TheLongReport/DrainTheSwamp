\section{The Pharisee Problem}

In their pursuit of moral purity, some within our movement have unintentionally become the very thing they claim to oppose. They have taken on the posture of the Pharisees—not the zeal of the prophets. While claiming to defend righteousness, they practice exclusion. While claiming to uphold virtue, they deliver condemnation. While claiming to model Christ, they embody His accusers.

They enforce a code of surface-level sanctity:
\begin{itemize}
    \item They gatekeep based on appearances, associations, and sanitized talking points.
    \item They demand perfection in others while excusing or ignoring their own failings.
    \item They burn bridges under the banner of ``righteous judgment,'' cutting off potential allies in the name of ``discernment.''
\end{itemize}

\textbf{You’re not wielding the whip like Jesus in the temple—you’re acting like the Pharisees who tried to block His ministry.}

Jesus did not surround Himself with saints. He chose a tax collector, a zealot, a doubter, and a traitor for His innermost circle. He dined with prostitutes and sinners. He healed the unclean. He offered living water to the scandal-ridden woman at the well. He broke bread with men the religious establishment considered unworthy. Why? Because He came to redeem, not to reject.

Yet today, too many Christian conservatives forget that distinction. They treat moral failure not as a reason to engage, but as a reason to exile. They forget that David—the man after God’s own heart—was an adulterer and a murderer. They forget that Paul—the apostle of grace—was a persecutor of Christians. They forget that Peter—our first Pope—denied Christ three times. Redemption was not a political liability in Scripture. It was the foundation of transformation.

\textbf{But in modern conservative politics, redemption is suspect. Repentance is viewed with skepticism. Grace is rarely granted.}

Instead, we demand repentance not for sin—but for association. ``Why didn’t you denounce him?'' ``Why did you speak at her event?'' ``Why did you vote with them on that bill?'' This is not accountability. It is Pharisaical inquisition—ritual purity masquerading as strategic discernment.

This mentality leads to internal tribunals instead of outreach. It leads to purity tests that not even the tester could pass. It leads to the assumption that proximity is complicity and that forgiveness equals compromise.

And worst of all, it leads to paralysis. Because no one is good enough. No one is untainted. No one is flawless. And so the circle of ``acceptable'' allies grows smaller and smaller—until the movement collapses in on itself.

\textbf{Politics is not a place for priests to demand confession from every passerby. It is a battleground where sinners with a spine must stand shoulder to shoulder to stop evil.}

The Church understands this. It does not demand moral perfection before Baptism. It demands surrender. It welcomes the fallen with repentance, not exclusion. The political right must rediscover that balance. Because what we are doing now—playing the role of moral gatekeeper while the Left conquers institution after institution—is not just spiritually misguided. It’s strategically suicidal.

There is nothing righteous about letting your nation fall because you could not bring yourself to work with an imperfect warrior. There is no virtue in losing to evil out of fear of contamination. And there is no Christlikeness in crucifying your own side.

The Pharisees memorized Scripture, tithed with precision, and obsessed over law. Yet Jesus called them whitewashed tombs—clean on the outside, dead within. Let that be a warning. If we care more about how someone looks than what they’re willing to fight for, we risk becoming beautiful tombs—polished, principled, and powerless.

\textbf{The Left does not care how moral their candidates are. They care how effective they are. We must reclaim virtue without weaponizing it, and wield strategy without abandoning it.}

Until then, we will continue to lose elections—and worse, we will continue to lose each other.