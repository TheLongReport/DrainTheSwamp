\section{The ``Burn It All Down'' Brigade}

There’s a growing faction within the Republican Party that doesn’t just want reform—it wants fire. Not controlled demolition, but total destruction. Not principled reconstruction, but an unrelenting purge. To them, everything that came before is corrupt. Everyone who participated is compromised. Every institution is irredeemable. If you weren’t born out of their specific political moment or ideological awakening, you are presumed guilty.

They view the GOP as a poisoned well—one so tainted that anyone who ever drank from it, worked within it, or even attempted to fix it from the inside is now an enemy of the cause. And they make it known.

\begin{itemize}
    \item ``If you worked with X, you’re disqualified.''
    \item ``If you endorsed Y five years ago, you’re a RINO.''
    \item ``If you don’t denounce Z on cue, you’re compromised.''
    \item ``If you attended the wrong event or shared the wrong stage, you're part of the swamp.''
\end{itemize}

This purist mentality is political suicide. It means we can never build coalitions. We can never evolve. We can never forgive. And we certainly can’t win. 

\textbf{Because winning requires addition. Burning it all down only leaves you with ashes.}

History is full of revolutions that consumed their own. The French Revolution ended with the guillotine slicing the heads of its own architects. The Bolsheviks devoured their allies before they turned on their enemies. The American Right must learn from these examples: when your first instinct is to purge rather than persuade, the movement becomes a snake eating its own tail.

This isn’t to say that corruption shouldn’t be confronted. It should. But not everyone who has ever worked within the party is corrupt. Not everyone who made a bad endorsement is compromised. Not everyone who once believed in the system is now part of the system. People evolve. Alliances shift. Some learn. Some grow. And some were never corrupt in the first place—they were simply navigating a broken structure with the best tools they had.

\textbf{If we cancel everyone with a past, we’ll cancel everyone with experience.}

What this faction misunderstands is that not all broken structures require a bomb. Some require courage. Some require wisdom. Some require the long, hard grind of reform. Not everyone who engages in that process is a sellout. Some are simply realists who know that you can’t rebuild a bridge while you’re setting it on fire.

The ``burn it down'' mentality also makes terrible assumptions:
\begin{itemize}
    \item That anyone not perfectly aligned is a threat.
    \item That distrust is more useful than discernment.
    \item That power is cleaner when it’s smaller.
\end{itemize}

But smaller does not mean purer. It means weaker. The goal is not to have the smallest, most righteous remnant shouting into the void. The goal is to govern—to win elections, change laws, preserve liberty, and defend civilization.

And that requires alliances. It requires nuance. It requires imperfect people. It requires a willingness to focus on who’s moving in the right direction, not just who checks the right boxes.

\textbf{If you spend all your time building gallows, don’t be surprised when you’re the only one left standing.}

We’ve seen this faction derail candidates, fracture conventions, sabotage party unity, and alienate potential voters—all in the name of cleansing the ranks. But what they fail to recognize is that purges do not build movements. They drain them. They silence energy. They turn off potential converts. And they hand victories to the very people we are supposed to be fighting.

The Left doesn’t fear the ``burn it all down'' crowd. They cheer it on. Because they know that when Republicans are too busy executing their own, Democrats win by default.

We need fire in our movement—but it must be a forge, not an inferno. A fire that tempers steel, not one that turns everything to ash. Anger without direction is just destruction. Principle without grace is just pride. And if we cannot learn to work with people who have been in the trenches—flawed though they may be—we will never retake the ground we've lost.

\textbf{Burning it all down may feel righteous—but it’s a shortcut to irrelevance.}