\section{Moral Absolutism vs. Political Warfare}

There is a vital difference between maintaining personal virtue and enforcing moral absolutism as a political strategy. The first is noble. The second is often suicidal.

You should absolutely hold your own standards. Your convictions matter. Your conscience is sacred. But when you enter the political arena—when you put on the armor of a statesman, a candidate, a strategist, or a party leader—you are no longer operating in a monastery. You’re on a battlefield. And battlefields don’t reward moral perfection. They reward clarity, courage, and coalition.

\begin{itemize}
    \item We are in a culture war—where every institution is tilted against us.
    \item We are in a spiritual war—where truth itself is under siege.
    \item But most immediately—we are in a political war. And wars are not won by isolating every potential ally over a past sin or flawed decision.
\end{itemize}

\textbf{Politics is not the priesthood. It’s triage. You take the people willing to fight with you, even if they need confession later.}

We must understand this: the moral demands of the Church and the strategic demands of the campaign trail are not the same thing. You do not win elections by enforcing excommunication. You win by building a majority. That does not mean sacrificing your soul. But it does mean being wise about how you build the army necessary to defeat evil.

The conservative movement has too often confused the role of the candidate with the role of the confessor. A priest can—and should—require contrition and penance. But a precinct delegate, a district chair, or a campaign manager must ask a different question: \textit{Is this person willing to fight for our values and defeat our opponents?}

If the answer is yes, then the mission takes precedence over the moral resume.

This is not a call to abandon principles. It’s a call to apply them with wisdom. Because if we make perfection the price of admission, we will never have a majority. If we demand total alignment before we collaborate, we will be left shouting from the sidelines while the country burns.

\textbf{We don’t need saints in every seat. We need fighters in every district.}

It is possible to uphold truth and still build bridges. It is possible to hold people accountable without canceling their entire future. And it is absolutely necessary to recognize that your opponent is not the imperfect person who shares 80\% of your views—it’s the radical progressive trying to erase your children’s future.

There’s a reason Christ said “be wise as serpents and innocent as doves.” Too many today act only as doves—and get slaughtered. Others act only as serpents—and become indistinguishable from the Left. But the true conservative—rooted in truth, disciplined in strategy—must do both. 

You can’t pray your way into a legislative majority. You can’t sermon your way to flipping school boards. And you can’t moralize your way to taking back the governor’s mansion. Those things require tactics. Coalitions. Planning. Votes.

\textbf{And votes come from people—flawed, messy, inconsistent people. Just like you. Just like me.}

We are not called to surround ourselves only with the pure. We are called to win so that we can preserve the space for purity, virtue, and truth to flourish.

So yes, carry your principles with pride. Refuse to lie. Refuse to sell out. But don’t confuse discernment with dismissal. Don’t reject a soldier because they’ve sinned. Don’t discard a candidate because they once stood in the wrong room. Don’t retreat into holy isolation while the Left marches unopposed into our schools, courthouses, and capitols.

\textbf{Moral absolutism feels righteous—but in the wrong hands, it’s just another form of surrender.}